\documentclass[11pt]{article}
\usepackage{amsmath}

\title{Calculating the Food Demand System}

\begin{document}
\maketitle

Assume the following form for the $i$th demand component (Here and
elsewhere, $i,j \in \lbrace s,n\rbrace$.):
\begin{equation}
  q_i = A_i w_i^{e_{ii}} w_j^{e_{ij}} x^{h_i},
\end{equation}
where $w_i = P_i/P_m$, $x = Y/P_m$, and $h_i$ is a function of $x$
only (i.e., $h_i = H_i(x)$).

Let $\varepsilon_{ij}$ be the price elasticities and $\eta_i$ be the
income elasticities.  Then,
\begin{equation}
  \label{eq:epsij}
  \varepsilon_{ij} = \frac{\partial \ln q_i}{\partial \ln w_j} =
  e_{ij} + w_j\left(\ln w_i \frac{\partial e_{ii}}{\partial w_j} + \ln
  w_j \frac{\partial e_{ij}}{\partial w_j}\right),
\end{equation}
and
\begin{equation}
  \label{eq:etai}
  \eta_i = \frac{\partial \ln q_i}{\partial \ln x} = h_i + \ln x
  \frac{\partial h_i}{\partial \ln x} + x\left(\ln w_i  \frac{\partial
    e_{ii}}{\partial x} + \ln w_j \frac{\partial e_{ij}}{\partial
    x}\right).
\end{equation}
Observe that the first two terms in equation~(\ref{eq:etai}) are what
we had previously calculated as the income elasticity, neglectng the
$x$ dependence of the $e_{ij}$ terms.  Accordingly, define
\begin{equation}
  \zeta_i = h_i + \ln x \frac{\partial h_i}{\partial \ln x}.
\end{equation}
This definition allows us to write the expressions that follow as a
sum of the old solution and the terms neglected by that solution.

We will try to write a solution where the Hicks elasticities
$\xi_{ij}$ are constant.  The elasticities must obey the Slutsky
equation
\begin{equation}
  \varepsilon_{ij} = \xi_{ij} - \alpha_j \eta_i,
\end{equation}
where $\alpha_j$ is the budget fraction, $\alpha_j = \frac{w_j
  q_j}{x}$.  Substituting the expressions for the elasticities from
equations~(\ref{eq:epsij}) and~(\ref{eq:etai}),
\begin{equation}
  e_{ij} + w_j\left(\ln w_i \frac{\partial e_{ii}}{\partial w_j} + \ln
  w_j \frac{\partial e_{ij}}{\partial w_j}\right) =
  \xi_{ij} - \alpha_j \left[\zeta_i + x\left(\ln
  w_i \frac{\partial e_{ii}}{\partial x} + \ln w_j \frac{\partial
    e_{ij}}{\partial x}\right)\right].
\end{equation}
Rearranging and collecting terms,
\begin{equation}
  \label{eq:PDE}
e_{ij} = \xi_{ij} - \alpha_j\zeta_i - \left[w_j \ln w_i \frac{\partial
    e_{ii}}{\partial w_j} + w_j \ln w_j \frac{\partial
    e_{ij}}{\partial w_j} + \alpha_j x \ln w_i \frac{\partial
    e_{ii}}{\partial x} + \alpha_j x \ln w_j \frac{\partial
    e_{ij}}{\partial x}\right].
\end{equation}
This can be rewritten in matrix form\footnote{The matrix transposes
  are necessary to ensure that the partial derivatives operate only on
  the matrix of $e_{ij}$ values.}:
%% Check that the matrix multiplies out to agree with the
%% equation above!
\begin{multline}
  \label{eq:PDEmat}
  \left[e_{ij}\right]^T = \left[\xi_{ij}-\alpha_j\zeta_i\right]^T - \\
  \left(
  \begin{bmatrix}
    \ln w_s \\
    \ln w_n
  \end{bmatrix}
  \begin{bmatrix}
    w_s \frac{\partial}{\partial w_s} & w_n \frac{\partial}{\partial w_n}
  \end{bmatrix} 
  + x
  \begin{bmatrix}
    \ln w_s \\
    \ln w_n
  \end{bmatrix}
  \begin{bmatrix}
    \alpha_s \frac{\partial}{\partial x} & \alpha_n
    \frac{\partial}{\partial x}
  \end{bmatrix}
  \right)^T
  \begin{bmatrix}
    e_{ss} & e_{sn} \\
    e_{ns} & e_{nn}
  \end{bmatrix}^T
\end{multline}

The first line of equation~(\ref{eq:PDEmat}) is what we had in the
original version of the model.  Because it neglects all the partial
derivatives of the price-term exponents $e_{ij}$ it is solvable
algebraically.  With the additional terms included
equation~(\ref{eq:PDEmat}) defines a set of partial differential
equations for the $e_{ij}$, which we could in principle solve with the
usual numerical methods.

From here I think we have a couple of options.
\begin{enumerate}
\item Attempt to solve equation~(\ref{eq:PDEmat}) in its full form
  numerically.  This is probably doable, but nasty.  We would probably
  have to precalculate the results and store them as a look-up table,
  and none of it would generalize to more than two components.
\item Attempt to simplify the equation by strategically neglecting
  certain terms.  For example, we could say that only the price
  derivatives of the form $\frac{\partial}{\partial w_i} e_{ij}$ are
  nonzero (i.e., the exponents in $q_s$ depend only on $w_s$, and the
  ones in $q_n$ depend only on $w_n$).  If we choose carefully we
  might be able to reduce this to a set of decoupled with just price
  and income as variables.
\item Include only the $\alpha_j \zeta_i$ terms, and assume that
  $\xi_{ij}$ adjusts accordingly (i.e., give up the assumption that
  $\xi_{ij}$ is constant.  This term is easiest computationally, but
  we would have to find a way to ensure that $\xi_{ss}$, $\xi_{nn}$,
  \emph{and} $\xi_{mm}$ are all positive, or at least ensure that any
  violations happened in price and income regimes that we aren't
  interested in.
\end{enumerate}
\end{document}

